% !TEX TS-program = pdflatexmk

\documentclass[modern]{aastex61}

\newcommand{\pystan}{\texttt{PyStan}}
\newcommand{\python}{\texttt{Python}}
\newcommand{\RRR}{\texttt{R}}
\newcommand{\stan}{\texttt{Stan}}

\begin{document}

\title{Astrophysics With Gravitational Wave Detections: Data Analysis}
\author{Will M. Farr}
\affil{Birmingham Institute for Gravitational Wave Astrophysics and School of Physics and Astronomy, University of Birmingham}
\email{w.farr@bham.ac.uk}

\begin{abstract}
  This series of three lectures is intended to give you an
  introduction to the data analysis techniques used in gravitational
  wave astrophysics and astrostatistics more generally.  
\end{abstract}

\section{Syllabus}

The goal of this course is to introduce you to some of the techniques
used in gravitational wave astrophysics and also to give you some
practice with the \emph{implementation} of statistical analyses.  To
that end, the ``lectures'' will hopefully be very interactive, with
opportunities to ask questions, work with your peers to develop code,
and see in real-time the results of your analyses.  Of course you do
not have to participate to this degree if you do not want;
nevertheless, I think you will get the most out of the school if you
do.

\emph{Before Monday afternoon}, please install a working \stan{}
implemtation \citep{Stan}.  I personally favour
\href{http://mc-stan.org}{\pystan} \citep{PyStan} on top of the
\href{https://www.continuum.io}{Anaconda} \python{} distribution, but
if you are more familiar with \RRR or the command line, go ahead.  If
you are using \pystan{}, you will know your installation is working
when you can run the ``eight schools'' example from the
\href{https://pystan.readthedocs.io/en/latest/getting_started.html}{\pystan{}
  manual}\footnote{For an interesting, intuitive description of
  hierarchical modelling---which will feature in Lecture 3---see
  \href{http://andrewgelman.com/2014/01/21/everything-need-know-bayesian-statistics-learned-eight-schools/}{here}.}.

Each lecture will come with reading material; some I will ask you to
familariase yourself with in advance, and some is just in case you are
interested to go further than I do in the lecture.  Please have a look
(it doesn't have to be extensive---just get familiar with the content,
or have questions about wherever you get stuck) at the material before
the lecture so we can discuss it straight away.

The material for the course, including this document, can be found on
\href{https://github.com/farr/GWDataAnalysisSummerSchool}{GitHub}.

\subsection{Lecture 1: Monday Afternoon.  The Bayesics}

It may seem simple, but this entire lecture will be concerned with the
material in \citet{Hogg2010}.  Please familarise yourself with it
before the lecture; we will be writing code to solve some of the
associated problems during the lecture.  

We will spend a bit of time talking about statistical notation and the
notion of conditional versus joint distributions, and then we will
dive in to fitting a line to data!  If you internalise all the
discussion from \citet{Hogg2010} you will be ready to build models
that are better than 90\% of the ``state-of-the-art'' papers on the
arXiv.  To explore the ideas in \citet{Hogg2010}, we will use the
modelling language \stan \citep{Stan}, so I will give a brief tutorial
on the structure of the language (the syntax should be familiar to
those who have programmed in \texttt{C} or similar languages before)
and how it is optimised for describing and solving the sorts of
statistical problems we are going to be talking about in this course.

Further reading (for those interested) includes
\href{http://dan.iel.fm/posts/fitting-a-plane/}{this blog post} by
Daniel Foreman-Mackey, \citet{Kelly2007}, \citet{Lieu2017}, \ldots.

\subsection{Lecture 2: Thursday Morning. Extracting Astrophysical
  Information from Gravitational Waves.}

This lecture will discuss the physics from
\citet{Abbott2017BasicPhysics} and the techniques described in
\citet{Veitch2015} for extracting information about gravitational wave
sources.  Please familarise yourself with these references before the
lecture.

We will explore how the various parameters of a merging compact object
system are encoded in the gravitational waveform
\citep{Abbott2017BasicPhysics}.  We will talk about the noise sources
in interferometric gravitational wave detectors and justify the
likelihood function given in \citet{Veitch2015}.  We will implement
this likelihood function in \stan{}, and use it to estimate the
parameters of GW150914 \citep{Abbott2016GW150914} from
\href{https://losc.ligo.org/events/GW150914/}{actual LIGO data}.

For further reading, consider \citet{Abbott2016PEPaper},
\citet{Abbott2016Wavelet}, \ldots.  

\subsection{Lecture 3: Friday Afternoon.  Population Modelling.}

This lecture will deal with three different topics in population
modelling.  First, we will derive the ``fundamental likelihood'' of
population modelling, following
\href{http://dan.iel.fm/posts/histogram1/}{this blog post} by Dan
Foreman-Mackey.  Then we will show how this likelihood generalises to
a \emph{mixture} model used to infer rates of black hole mergers in
\citet{Abbott2016Rate}; we will write some \stan{} code to infer rates
of binary black hole mergers and apply it to real data from that
paper.  Finally, we will discuss \emph{model selection} in a
gravitational waves context, following \citet{Farr2017}.  Please
familiarise yourself with these references before the lecture.

For further reading, consider \citet{Abbott2016RatesSupplement},
\citet{Foreman-Mackey2014}, \citet{Farr2015}, \citet{Loredo1995},
\citet{Farr2011}, \ldots.

\bibliography{syllabus}

\end{document}